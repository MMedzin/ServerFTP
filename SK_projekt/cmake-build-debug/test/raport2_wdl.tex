\documentclass[12pt]{article} 
\usepackage{amsmath}  
\usepackage{amssymb,amsfonts,amsthm}
\usepackage[polish]{babel} 
\usepackage[UTF8]{inputenc} 
\usepackage[T1]{fontenc}
\usepackage{graphicx}
\usepackage{array}
\usepackage[bottom=1.1in,top=1.1in, left=1.0in, right=1.0in]{geometry}
\usepackage{amsmath, amsfonts, amssymb, amsthm}
\usepackage{mathtools}  
\usepackage{hyperref}
\usepackage{pgfplots}
\usepackage{tikz}
\usetikzlibrary{shapes.geometric, arrows}

\title{{\bfseries Raport z zadania domowego nr 2}\\ {\Large z wprowadzenia do logiki}}
\author{Maria Mikołajczak, Weronika Piechaczyk \\( grupa nr 3)}
\tikzstyle{startstop} = [rectangle, rounded corners, minimum width=1.5cm, minimum height=1cm,text centered, draw=black]
\tikzstyle{box} = [rectangle,rounded corners, text width=1em, minimum height=1cm,text centered, draw=black]
\tikzstyle{arrow}=[->,>=stealth]
\begin{document}
	\maketitle
	\section{Zadanie z części pierwszej}
	\begin{itemize}
		\item	Wylosowane zadanie: zadanie nr 3
		\item Treść zadania: \vspace{0.5em} \\ Wykonaj kolejno kontrapozycję, konwersję i obwersję i zdania: \\
		\textit{Każdy kognitywista jest urwisem.}
		\item Rozwiązanie:\\
		 Przyjmujemy oznaczenia:
		\begin{itemize}
			\item S - kognitywista
			\item P - urwis
		\end{itemize}
	Zdanie ,,\textit{Każdy kognitywista jest urwisem}'' jest zdaniem ogólnotwierdzącym, więc możemy zastąpić je schematem SaP. 
		\begin{enumerate}
			\item \textbf{Kontrapozycja}:
			
			$S$a$P$ $\rightarrow$ $P'$a$S'$\\
			Każdy kognitywista jest urwisem. $\rightarrow$ Każdy nie-urwis jest nie-kognitywistą.
			\item \textbf{Konwersja:}
			
			$P'$a$S'$ $\rightarrow$ $S'$i$P'$ \\
			Każdy nie-urwis jest nie-kognitywistą. $\rightarrow$ Niektórzy nie-kognitywiści są nie-urwisami.
			\item \textbf{Obwersja}:
			
			$S'$i$P'$ $\rightarrow$ $S'$o$P''$\\
			Niektórzy nie-kognitywiści są nie-urwisami. $\rightarrow$ Niektórzy nie-kognitywiści nie są urwisami.
			
		\end{enumerate}
	\end{itemize}
	\section{Zadanie z części drugiej}
	\begin{itemize}
		\item	Wylosowane zadanie: zadanie nr 4
		\item Treść zadania: \vspace{0.5em} \\ Jakie wnioski można wyprowadzić z następującego zbioru przesłanek? Czy można wyciągnąć
		jakikolwiek? Uzasadnij swoją odpowiedź i zaproponuj interpretację (przedstaw konkretny przykład
		zbudowany na schemacie rozważanego sylogizmu).

		
		Przesłanki: $P$a$M$, $S$o$M$.

		\item Rozwiązanie: 
		
		Poniższe kroki reprezentują nasz proces myślowy przy poszukiwaniu potencjalnych wniosków z danego zbioru przesłanek.
		\begin{enumerate}
			\item \textbf{Wyznaczenie terminu średniego.}
			
			Terminem średnim jest nazwa $M$, ponieważ występuje w obu przesłankach.
			Wiemy więc, że termin ten nie może wystąpić we wniosku.
			
			
			\item \textbf{Sprawdzenie, czy termin średni jest rozłożony w co najmniej jednej z przesłanek. }
			
			
			Terminy rozłożone zostały oznaczone przez podkreślenie:
			$\underline{P}$a$M$, $S$o$\underline{M}$.
			Termin średni jest rozłożony w przesłance $S$o$M$, a więc spełniony jest pierwszy warunek poprawności sylogizmu i możemy podjąć dalszą analizę tego przypadku.
			
			\item \textbf{Sprawdzenie, czy co najmniej jedna przesłanka jest zdaniem twierdzącym.}
			
			Przesłanka $P$a$M$ jest zdaniem twierdzącym, więc ten warunek poprawności także jest spełniony i możemy podjąć dalszą analizę.
			
			\item \textbf{Określenie jakiego typu zdaniem może być wniosek.}
			
			Wniosek musi być zdaniem przeczącym, ponieważ jedna z przesłanek jest takim zdaniem- $S$o$M$.
			
			\item \textbf{Określenie możliwej figury sylogizmu.}
			
			W obu przesłankach termin średni występuje w roli orzecznika, więc możliwe są dwa przypadki figury II:
			\begin{center}
				
						\begin{tabular}{c}
							$PM$ \\ 
							$SM$ \\ 
							\hline 
							$SP$ \\ 
						\end{tabular} 
											\begin{tabular}{c}
						$SM$ \\ 
						$PM$ \\ 
						\hline 
						$PS$ \\ 
					\end{tabular} 
					\end{center}
			
			W pierwszym przypadku nazwa $P$ jest terminem większym, a w przypadku drugim nazwa $S$.
			\item \textbf{Zbadanie poprawności potencjalnych wniosków.}
			
			Z powyższej analizy wynika, że możliwe są cztery sytuacje:
			
			\begin{enumerate}
			\item Wniosek w postaci zdania ogólnoprzeczącego, termin większy $P$. \\
			\begin{center}
				
				\begin{tabular}{c}
					$P$a$M$ \\ 
					$S$o$M$ \\ 
					\hline 
					$S$e$P$ \\ 
				\end{tabular} 
			\end{center}
		
		Aby sprawdzić, czy taki sylogizm jest poprawny musimy zbadać, czy terminy rozłożone we wniosku są rozłożone także w przesłankach:
		\begin{center}
			
			\begin{tabular}{c}
				$\underline{P}$a$M$ \\ 
				$S$o$\underline{M}$ \\ 
				\hline 
				$\underline{S}$e$\underline{P}$ \\ 
			\end{tabular} 
		\end{center}
	
	Powyższy sylogizm nie jest poprawny, ponieważ termin mniejszy $S$ rozłożony jest we wniosku, a nie jest rozłożony w przesłankach. Odrzucamy więc możliwość pierwszą (a).
	\item Wniosek w postaci zdania ogólnoprzeczącego, termin większy $S$. \\
	\begin{center}
		
		\begin{tabular}{c}
			$P$a$M$ \\ 
			$S$o$M$ \\ 
			\hline 
			$P$e$S$ \\ 
		\end{tabular} 
	\end{center}
	
	W tym przypadku będzie taki sam problem jak w sytuacji a, ponieważ w zdaniu ogólnoprzeczącym rozłożone są obie nazwy, a przesłankach rozłożona jest jedynie nazwa $P$:
	\begin{center}
		
		\begin{tabular}{c}
			$\underline{P}$a$M$ \\ 
			$S$o$\underline{M}$ \\ 
			\hline 
			$\underline{P}$e$\underline{S}$ \\ 
		\end{tabular} 
	\end{center}
	Odrzucamy więc także tę możliwość.
		\item Wniosek w postaci zdania szczegółowoprzeczącego, termin większy $P$. \\
		\begin{center}
			
			\begin{tabular}{c}
				$P$a$M$ \\ 
				$S$o$M$ \\ 
				\hline 
				$S$o$P$ 
			\end{tabular} 
		\end{center}
	Analogicznie sprawdzamy terminy rozłożone we wniosku:
	\begin{center}
	\begin{tabular}{c}
			$\underline{P}$a$M$ \\ 
		$S$o$\underline{M}$ \\ 
		\hline 
		$S$o$\underline{P}$ 
	\end{tabular} 
\end{center}
W przypadku zdania szczegółowoprzeczącego rozłożony jest termin większy $P$. Jest on także rozłożony w przesłance $P$a$M$, a więc powyższy sylogizm jest poprawny.
\item Wniosek w postaci zdania szczegółowoprzeczącego, termin większy $S$. \\
\begin{center}
	
	\begin{tabular}{c}
		$P$a$M$ \\ 
		$S$o$M$ \\ 
		\hline 
		$P$o$S$ 
	\end{tabular} 
\end{center}
Analogicznie sprawdzamy terminy rozłożone we wniosku:
\begin{center}
	\begin{tabular}{c}
		$\underline{P}$a$M$ \\ 
		$S$o$\underline{M}$ \\ 
		\hline 
		$P$o$\underline{S}$ 
	\end{tabular} 

Taki sylogizm jest niepoprawny, ponieważ termin $S$ jest rozłożony we wniosku, a nie jest rozłożony w przesłankach. Odrzucamy tę opcję.
\end{center}
			\end{enumerate}
		\item \textbf{Podsumowanie.}
		
		Powyższa analiza wykazała, że jedynym wnioskiem, który można wyciągnąć z danego zbioru przesłanek przy zachowaniu poprawności sylogizmu jest zdanie o schemacie $S$o$P$.
		\item \textbf{Przykładowa interpretacja. }
		
		\begin{itemize}
			\item Przesłanka nr 1: $P$a$M$ - Każdy dziobak jest ssakiem.
			\item Przesłanka nr 2: $S$o$M$ - Niektóre zwierzęta nie są ssakami.
			\item Wniosek: $S$o$P$- Niektóre zwierzęta nie są dziobakami.
		\end{itemize}
		\end{enumerate}
	\end{itemize}
	\section{Zadanie z części trzeciej}
	\begin{itemize}
		\item Treść zadania: \vspace{0.5em} \\ Zbuduj schematy zdań (w KRP) z przykładu wymyślonego w poprzednim zadaniu (wskazując
		interpretację poszczególnych stałych pozalogicznych).
		\item Rozwiązanie: \\
		\begin{itemize}
			\item $P^1_1$- predykat jednoargumentowy ,,być ssakiem''
			\item $P^1_2$ - predykat jednoargumentowy ,,być dziobakiem''
			\item $P^1_3$- predykat jednoargumentowy ,,być zwierzęciem''
			\item $x$ - zmienna indywiduowa
		\end{itemize}
		\begin{enumerate}
			\item \textbf{\textit{Każdy dziobak jest ssakiem.}}
			\[ \forall x (P^1_2(x) \rightarrow P^1_1(x)) \]
			\item \textbf{\textit{Niektóre zwierzęta nie są ssakami.}}
			\[ \exists x (P^1_3(x) \wedge \neg P^1_1(x)) \]
			\item \textbf{\textit{Niektóre zwierzęta nie są dziobakami.}}
			\[\exists x (P^1_3(x) \wedge \neg P^1_2(x) ) \]			
		\end{enumerate}
		
	
		
	\end{itemize}
\end{document}